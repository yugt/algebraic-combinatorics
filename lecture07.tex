% Friday 1.19
\section{Expander Graphs}
Vaguely speaking, an expander graph is a sparse graph (\textit{i.e.}, it has few edges) which is nonetheless highly connected.
Expander graphs have applications to
\begin{itemize}
\item constructing error-correcting codes
\item derandomize algorithms
\item serve methods in number theory
\item hyperbolic manifolds
\end{itemize}

\begin{definition}[edge expansion ratio]
Lety \(G\) be a graph.
For \(S\subseteq V(G)\) denote the set of edges with one endpoint in \(S\) and one endpoint not in \(S\) by \(\partial S\).
Define the edge expansion ratio
\[ h(G)\coloneqq \min_{\substack{S\subseteq V(G)\\|S|\leq|V(G)|/2}} \frac{|\partial S|}{|S|} \]
\end{definition}
\begin{definition}
A sequence of \(k\)-regular graphs \((G_n)_{n\in\N}\) is called a family of expander graphs if
\[ |V(G_0)|<|V(G_1)|<\cdots \]
and there exists \(\varepsilon>0\) such that \(h(G_n)\geq0,\forall n\in\N\).
\end{definition}
\begin{remark}
It's relatively easy to show that such families exist by probabilistic arguments, but it's relatively difficult to construct them explicitly.
\end{remark}
\begin{theorem}
Suppose \(G\) is connected and \(k\)-regular so \(\lambda_1(G)=k\), then
\[ \frac{k-\lambda_2(G)}{2}\leq h(G)\leq \sqrt{2k(k-\lambda_2(G))} \]
\end{theorem}
Therefore a family \((G_n)_{n\in\N}\) of \(k\)-regular graphs with \(|V(G_0)|<|V(G_1)|<\cdots\) forms a family of expander graphs \iff there exists \(\varepsilon'>0\) such that
\[ k=\lambda_2(G_n)\geq\varepsilon' \quad \forall n\in\N \]
The spectral gap \(k-\lambda_2(G)\) or sometimes \( \min\{|k-\lambda_2{G}|,k-|\lambda_n(G)|\} \) measures how rapidly a random walk on \(G\) mixes.



\chapter[Tilings \& Spanning Trees]{Tilings, Spanning Trees and Electrical Networks}

\begin{itemize}
\item motivation (and some results) come from Physics, Chemistry and Computer Science.
\end{itemize}

\section{Tilings and Perfect Matchings}
We'll prove the following theorems.
\begin{theorem}[Kasteleyn (1961)]
Let \(m,n\in\N\) be such that \(mn\) is even.
The number of domino tilings of an \(m\times n\) board equals
\[ T(m,n)=\prod_{j=1}^{m}\prod_{k=1}^n \left(4\cos^2\left(\frac{\pi j}{m+1}\right) + 4\cos^2\left(\frac{\pi k}{n+1}\right)\right)^{\frac{1}{4}} \]
\end{theorem}
\begin{example}
\(\pi(2,3)\) has 3 tilings.
\begin{figure}[h]
    % Add graph
    \caption{Three tilings of \(\pi(2,3)\)}
\end{figure}
\end{example}
\begin{remark}
If \(m,n\) both odd, there's no domino tilings of an \(m\times n\) board.
\end{remark}


\begin{definition}[Perfect matching]
A \emph{perfect matching} of a graph \(G\) is a subset of its edges which meets every vertex exactly once.
\end{definition}
\begin{remark}
\begin{flushleft}\end{flushleft}
\begin{itemize}
    \item Only graphs with even number of vertices can have a perfect matching.
    \item Note that domino tiling of an \(m\times n\) board correspond to perfect matchings of the \(m\times n\) grid \(P_m\square P_n\).
\end{itemize}
\end{remark}
\begin{example}
\(K_4\) has 3 perfect matchings. Each of them are rotated from anther.
\begin{figure}[ht]
    % Add graph
    \caption{Three perfect matchings of \(K_4\)}
\end{figure}
\end{example}
We will present a general method for conducting perfect matchings of certain graphs due to Kasteleyn, called Pfaffian method.

\section{Skew-Symmetric Matrices}
\begin{definition}[skew-symmetric matrix]
A matrix \(A\in\R^{n\times n}\) is called \emph{skew symmetric} if
\[ A^T=-A. \]
\end{definition}
\begin{remark}
Note that for skew-symmetric \(A\),
\[ \det(A)=\det(A^T)=\det(-A)=(-1)^n\det(A) \]
Therefore if \(n\) is odd, \(\det(A)=0\).
However, if \(n\) is even, \(\det(A)\) is a perfect square in the entries of \(A\).
\end{remark}
\begin{example}
\[ \det\begin{bmatrix}0 & a & b & c \\ -a & 0 & d & e \\ -b & -d & 0 & f \\ -c & -e & -f & 0 \end{bmatrix}=(af-be+cd)^2=\text{Pfaffian}^2 \]
\end{example}
\begin{definition}[Pfaffian]
Let \(A\in\R^{2n\times 2n}\) be skew-symmetric. We define the \emph{Pfaffian} of \(A\)
\[ \mathrm{pf}(A)\coloneqq \sum_{\substack{\text{perfect matching of }K_{2n}\\ M=\{\{i_1,j_1\},\cdots,\{i_n,j_n\}\} }} \mathrm{sgn}\begin{pmatrix} 1&2&3&4&\cdots&2n-1&2n\\ i_1&j_1&i_2&j_2&\cdots&i_n&j_n\end{pmatrix}\prod_{r=1}^n A_{i_r,j_r} \]
\end{definition}
\begin{example}
Let \(A\in\R^{4\times4}\) be skew-symmetric, then
\begin{align*}
    \mathrm{pf}(A)&=\mathrm{sgn}\begin{pmatrix} 1&2&3&4\\ 1&2&3&4\end{pmatrix}A_{1,2}A_{3,4}\\
    &+\mathrm{sgn}\begin{pmatrix} 1&2&3&4\\ 1&3&2&4\end{pmatrix}A_{1,3}A_{2,4}\\
    &+\mathrm{sgn}\begin{pmatrix} 1&2&3&4\\ 1&4&2&3\end{pmatrix}A_{1,4}A_{2,3}\\
    &=A_{1,2}A_{3,4}-A_{1,3}A_{2,4}+A_{1,4}A_{2,3}
\end{align*}
\end{example}
\begin{remark}
Why is the Pfaffian well-defined?
\begin{enumerate}[(a)]
    \item the M-term does not change if we swap the order of a given edge
    \[ \{i_r,j_r\} \longleftrightarrow \{j_r,i_r\} \]
    \begin{itemize}
        \item \(\mathrm{sgn}\begin{pmatrix} \cdots&2r-1&2r&\cdots \\ \cdots&i_r&j_r&\cdots \end{pmatrix}=-\mathrm{sgn}\begin{pmatrix} \cdots&2r-1&2r&\cdots \\ \cdots&j_r&i_r&\cdots \end{pmatrix} \)
        \item \(A_{i_r,j_r}=-A_{j_r,i_r}\)
    \end{itemize}
    \item the M-term does not change if we swap two consequtive edges
    \[ \{i_r,j_r\} \longleftrightarrow \{i_{r+1},j_{r+1}\} \]
    \begin{itemize}
    \item \(\mathrm{sgn}\begin{pmatrix} \cdots&2r-1&2r&2r+1&2r+2&\cdots \\ \cdots&i_r&j_r&i_{r+1}&j_{r+1}&\cdots \end{pmatrix} \)
    \item unchanged
    \begin{question}
        This part is not clear enough.
    \end{question}
\end{itemize}
\end{enumerate}
So we can compute
\[ \mathrm{pf}(A)=\underbrace{\frac{1}{2^nn!}}_{\substack{\text{there are }\frac{(2n)!}{2^nn!}\\ \text{ perfect matchings}}}\sum_{\pi\in\mathfrak{S}_{2n}} \mathrm{sgn}(\pi)\prod_{i=1}^{2n}A_{\pi(2i-1),\pi(2i)} \]
\end{remark}