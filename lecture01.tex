\begin{itemize}
\item Linear algebra
\item Group theory (Cayley graphs, Dynkin diagrams)
\end{itemize}

\section{Eigenvalues}

\begin{definition}[Adjacency matrix]
Let \(G=(V,E)\) be a finite graph.
The \emph{adjacency matrix} \( A(G)=\{0,1\}^{V\times V} \) is defined by
\[ A(G)_{v,w}=\begin{cases} 1 & \text{if } v\sim w \\ 0 & \text{otherwise} \end{cases} \]
\end{definition}

Recall the characteristic polynomial of a square matrix \(M\) over \C
\[ \phi_M(t)\coloneqq \det(tI-M). \]


\begin{definition}[Characteristic polynomial]
Let \(G=(V,E)\) be a finite graph.
The \emph{Characteristic polynomial} is defined by
\[ \phi_G(t)\coloneqq \phi_{A(G)}(t), \]
and call the zeros (with multiplicitier) the eigenvalues of \(G\).
\end{definition}

\begin{definition}[Spectrum]
The \emph{spectrum} of \(G\) is the multiset of its eigenvalues.
\end{definition}

\begin{example}[spectrum]
\begin{align*}
A(G)&=\begin{bmatrix} 0 & 1 \\ 1 & 0 \end{bmatrix}\\
\phi_G(t)&=\det\begin{pmatrix} t & -1 \\ -1 & t \end{pmatrix}=-t^2-1\\
\text{spectrum}&: \{1,-1\}
\end{align*}
\end{example}

\begin{remark}
Eigenvalues provide information about the connectivity of the graph.
\end{remark}


\begin{theorem}
Let \(A\in\R^{n\times n}\) be symmetric, then \(A\) \emph{orthogonally diagonalizable} over \R, \text{i.e.},
there exists an orthonormal basis
\[ v^{(1)}, v^{(2)},\cdots,v^{(n)}\in\R^n \]
of eigenvectors of \(A\) corresponding to real eigenvalues \(\lambda_1,\lambda_2,\cdots,\lambda_n\in\R\), we have
\[A=\sum_{i=1}^{n}\lambda_i v^{(i)}v^{(i)T} \]
\end{theorem}

Therefore the eigenvalues of any graph \(G\) are all real and we'll denote then
\[ \lambda_1(G)\geq\lambda_2(G)\geq\cdots\geq\lambda_n(G), \]
where \(n=|V(G)|\).

\begin{theorem}[Perron-Forbenius]
If a matrix \(A\in\R^{n\times n}\) has nonnegative entries, then the spectral radius of \(A\) (\textit{i.e.}, the maximum magnitude over all complex eigenvalues of A) is an eigenvalue of \(A\), corresponding to an eigenvector in \(\R_{\ge0}^n\).
\end{theorem}
Therefore for any graph \(G\), \(\lambda_1(G)\) is the spectral radius and corresponds to an eigenvector with nonnegative entries.
Perron-Forbenius also implies if \(G\) is connected, then \(\lambda_1(G)\) has multiplicity 1.


\begin{definition}[disjoint union]
If \(G=(V,E)\), \(G'=(V',E')\) are graphs, their \emph{disjoint union} is the graph
\[ G\sqcup G'=(V\sqcup V', E\sqcup E') \]
and
\[ A(G\sqcup G)=\begin{bmatrix} A(G) & 0 \\ 0 & A(G') \end{bmatrix} \]
\end{definition}
\begin{remark}
Spectrum doesn't detect if the graph is connected.
\end{remark}
\begin{example}
\begin{align*}
G&: V=\{1,2\}, E=\{\{1,2\}\}\\
\text{spectrum}&: \{1,-1\}\\
G\sqcup G&: V=\{1,2,3,4\}, E=\{\{1,2\},\{3,4\}\}\\
\text{spectrum}&: \{1,1,-1,-1\}
\end{align*}
\end{example}

\begin{remark}
Note that the spectrum of \(G\sqcup G'\) is the multiset union of the spectra of \(G\) and \(G'\).
\end{remark}


