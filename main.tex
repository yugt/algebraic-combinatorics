		% This is the repository for Math 566 Algebraic Combinatorics Lecture Notes.

% The materials are voluntarily typed, and the professor are not likely to proofread them.

% If you are not familiar with LaTeX or don't want to type, you can click on the PDF on the black bar on the top of the page to download the PDF file.

% If you are willing to contribute by typing some lecture notes, click on the PROJECT icon on the black bar on the top of the page, and a side bar will occur on the left. Click on the corresponding lecture number and edit the file. You can refer to lecture01.tex as a sample.

% If you have some questions on the course materials, you can type the question in the question environment. It will be highlighted in red.

% If you have the answer to the questions, you can type the answer in the answer environment. It will be highlighted in green.

% If you are familiar with git commands, you can clone the repository and edit offline, and pull and push to synchronize.

\documentclass[12pt,oneside]{book}
\usepackage[letterpaper, margin=1in, headheight=105pt]{geometry}
\usepackage{kbordermatrix}
\usepackage{subcaption}
\usepackage{enumerate}
\usepackage{mathtools}
\usepackage{listings}
\usepackage{calligra}
\usepackage{fancyhdr}
\usepackage{titlesec}
\usepackage{graphicx}
\usepackage{amsmath}
\usepackage{amssymb}
\usepackage{caption}
\usepackage{titling}
\usepackage{comment}
\usepackage{xparse}
\usepackage{amsthm}
\usepackage[dvipsnames]{xcolor}
\usepackage[colorlinks=true,
			linkcolor=WildStrawberry,
			anchorcolor=red,
			citecolor=green,
			urlcolor=RoyalBlue,
			filecolor=brown,
			menucolor=pink]{hyperref}
\usepackage[linesnumbered,
			boxed,
			algochapter,
			algo2e,
			lined]{algorithm2e}
\usepackage[numbered,
			framed]{matlab-prettifier}
\usepackage[backend=biber,backref=true]{biblatex}
\usepackage[titletoc]{appendix}
\addbibresource{reference.bib}

%% tikz setting for graph drawing
\usepackage{tikz}
\usetikzlibrary{positioning}
\tikzset{plain/.style={circle,fill=blue!20,draw,minimum size=0.5cm,inner sep=0pt},
}
\usetikzlibrary{arrows}
\usepackage{tkz-graph}

\makeatletter
\@addtoreset{chapter}{part}
\makeatother

\title{\textbf{Math 566 Algebraic Combinatorics}}
\author{\calligra{Steven Karp}}
\date{\today}


\newcommand{\Z}{\ensuremath{\mathbb{Z}}}
\newcommand{\R}{\ensuremath{\mathbb{R}}}
\newcommand{\N}{\ensuremath{\mathbb{N}}}
\newcommand{\Q}{\ensuremath{\mathbb{Q}}}
\newcommand{\C}{\ensuremath{\mathbb{C}}}
\DeclareMathOperator*{\E}{\ensuremath{\mathbb{E}}}
\DeclareMathOperator*{\tr}{\ensuremath{\mathrm{tr}}}
\DeclareMathOperator*{\rank}{\ensuremath{\mathrm{rank}}}
\renewcommand{\iff}{\ifmmode \iff \else \text{if and only if} \fi}
\DeclarePairedDelimiter\ceil{\lceil}{\rceil}
\DeclarePairedDelimiter\floor{\lfloor}{\rfloor}


\newtheorem{theorem}{Theorem}[chapter]
\newtheorem{lemma}{Lemma}[chapter]
\newtheorem{proposition}{Proposition}[chapter]
\newtheorem{corollary}{Corollary}[chapter]
\theoremstyle{definition}
\newtheorem{definition}{Definition}[chapter]
\newtheorem{example}{Example}[chapter]
\theoremstyle{remark}
\newtheorem*{remark}{Remark}

\newtheoremstyle{ques}{}{}{\color{red}}{}{\bfseries\color{red}}{:}{ }{}
\theoremstyle{ques}
\newtheorem*{question}{Question}

\newtheoremstyle{ans}{}{}{\color{blue}}{}{\bfseries\color{blue}}{:}{ }{}
\theoremstyle{ans}
\newtheorem*{answer}{Answer}

\graphicspath{{./graphics/}}

\setlength{\algomargin}{2em}
\lstset{xleftmargin=.1\textwidth, xrightmargin=.1\textwidth}

\pagestyle{fancy}


\begin{document}

\chapter{Algebraic Graph Theory}

\begin{itemize}
\item Linear algebra
\item Group theory (Cayley graphs, Dynkin diagrams)
\end{itemize}

\section{Eigenvalues}

\begin{definition}[Adjacency matrix]
Let \(G=(V,E)\) be a finite graph.
The \emph{adjacency matrix} \( A(G)=\{0,1\}^{V\times V} \) is defined by
\[ A(G)_{v,w}=\begin{cases} 1 & \text{if } v\sim w \\ 0 & \text{otherwise} \end{cases} \]
\end{definition}

Recall the characteristic polynomial of a square matrix \(M\) over \C
\[ \phi_M(t)\coloneqq \det(tI-M). \]


\begin{definition}[Characteristic polynomial]
Let \(G=(V,E)\) be a finite graph.
The \emph{Characteristic polynomial} is defined by
\[ \phi_G(t)\coloneqq \phi_{A(G)}(t), \]
and call the zeros (with multiplicitier) the eigenvalues of \(G\).
\end{definition}

\begin{definition}[Spectrum]
The \emph{spectrum} of \(G\) is the multiset of its eigenvalues.
\end{definition}

\begin{example}[spectrum]
\begin{align*}
A(G)&=\begin{bmatrix} 0 & 1 \\ 1 & 0 \end{bmatrix}\\
\phi_G(t)&=\det\begin{pmatrix} t & -1 \\ -1 & t \end{pmatrix}=-t^2-1\\
\text{spectrum}&: \{1,-1\}
\end{align*}
\end{example}

\begin{remark}
Eigenvalues provide information about the connectivity of the graph.
\end{remark}


\begin{theorem}
Let \(A\in\R^{n\times n}\) be symmetric, then \(A\) \emph{orthogonally diagonalizable} over \R, \text{i.e.},
there exists an orthonormal basis
\[ v^{(1)}, v^{(2)},\cdots,v^{(n)}\in\R^n \]
of eigenvectors of \(A\) corresponding to real eigenvalues \(\lambda_1,\lambda_2,\cdots,\lambda_n\in\R\), we have
\[A=\sum_{i=1}^{n}\lambda_i v^{(i)}v^{(i)T} \]
\end{theorem}

Therefore the eigenvalues of any graph \(G\) are all real and we'll denote then
\[ \lambda_1(G)\geq\lambda_2(G)\geq\cdots\geq\lambda_n(G), \]
where \(n=|V(G)|\).

\begin{theorem}[Perron-Forbenius]
If a matrix \(A\in\R^{n\times n}\) has nonnegative entries, then the spectral radius of \(A\) (\textit{i.e.}, the maximum magnitude over all complex eigenvalues of A) is an eigenvalue of \(A\), corresponding to an eigenvector in \(\R_{\ge0}^n\).
\end{theorem}
Therefore for any graph \(G\), \(\lambda_1(G)\) is the spectral radius and corresponds to an eigenvector with nonnegative entries.
Perron-Forbenius also implies if \(G\) is connected, then \(\lambda_1(G)\) has multiplicity 1.


\begin{definition}[disjoint union]
If \(G=(V,E)\), \(G'=(V',E')\) are graphs, their \emph{disjoint union} is the graph
\[ G\sqcup G'=(V\sqcup V', E\sqcup E') \]
and
\[ A(G\sqcup G)=\begin{bmatrix} A(G) & 0 \\ 0 & A(G') \end{bmatrix} \]
\end{definition}
\begin{remark}
Spectrum doesn't detect if the graph is connected.
\end{remark}
\begin{example}
\begin{align*}
G&: V=\{1,2\}, E=\{\{1,2\}\}\\
\text{spectrum}&: \{1,-1\}\\
G\sqcup G&: V=\{1,2,3,4\}, E=\{\{1,2\},\{3,4\}\}\\
\text{spectrum}&: \{1,1,-1,-1\}
\end{align*}
\end{example}

\begin{remark}
Note that the spectrum of \(G\sqcup G'\) is the multiset union of the spectra of \(G\) and \(G'\).
\end{remark}



\begin{example}
Spectrum does not tell connectivity. For example, the following two graphs have the same spectrum $(-2,2,0,0,0)$.
\begin{figure}[tbp]
\centering
\begin{tikzpicture}
%% disconnected graph
\node[plain](1) {};
\node[plain](2) [right = 1cm of 1]  {};
\node[plain](3) [above = 1cm of 1] {};
\node[plain](4) [left = 1cm of 1] {};
\node[plain](5) [below = 1cm of 1] {};
\path[draw,thick]
(2) edge node {} (3)
(3) edge node {} (4)
(4) edge node {} (5)
(5) edge node {} (2);
%% connected graph
\begin{scope}[xshift=6cm]
\node[plain](1) {};
\node[plain](2) [right = 1cm of 1]  {};
\node[plain](3) [above = 1cm of 1] {};
\node[plain](4) [left = 1cm of 1] {};
\node[plain](5) [below = 1cm of 1] {};
\path[draw,thick]
(1) edge node {} (3)
(1) edge node {} (4)
(1) edge node {} (5)
(1) edge node {} (2);
\end{scope}
\end{tikzpicture}
\label{fig:spectrum_connectivity_example}
\caption{Two graphs with the same spectrum}
\end{figure}
\end{example}
\section{Bipartite Graphs}
Recall graph is bipartite with biparts \(x,y\) \iff
\[ A(G)=\begin{bmatrix}0 & B \\ B^T & 0\end{bmatrix} \]

\begin{lemma}
Let \(B\) be an \(n\times n\) matrix such that \(B^TB\) has nonzero eigenvalues \(\lambda_1,\cdots,\lambda_r\) (with multiplicities), then the nonzero eigenvalues of 
\[ A=\begin{bmatrix}0 & B \\ B^T & 0\end{bmatrix}\in\R^{(m+n)\times(m+n)} \]
are precisely \( \sqrt{\lambda_1},-\sqrt{\lambda_1},\cdots,\sqrt{\lambda_r},-\sqrt{\lambda_r} \) (with multiplicities).
\end{lemma}
\begin{remark}
The eigenvalues of \(B^TB\) are real (from symmetry) and nonnegative since if \(B^T Bx=\lambda x\) (we assume \(||x||=1\))
\[ \lambda =\lambda x^T x= x^T (\lambda x)= x^T(B^T Bx) =(Bx)^T(Bx)=\langle Bx,Bx\rangle\geq0. \]
\end{remark}
\begin{proof}
Note that
\[ \begin{bmatrix}t I_m & -B \\ -B^T & t I_n\end{bmatrix}\begin{bmatrix} I_m & B \\ 0 & t I_n\end{bmatrix}=\begin{bmatrix}t I_m & 0 \\ -B^T & t^2 I_n-B^TB \end{bmatrix}. \]
Take determinant gives
\[ \phi_A(t) t^n=t^m\phi_{B^TB}(t^2) \]
\end{proof}
\begin{question}
Why does the matrix on RHS have the determinant \(t^m\phi_{B^T B}(t)\)?
Particularly, why is
\[ \det\left(t^2I_n-B^T B\right)=\phi_{B^T B}\left(t^2\right)? \]
\end{question}
\begin{answer}
By definition, $\phi_{B^T B}(x)=\det(x I_n-B^T B)$. Therefore, $\phi_{B^T B}(t^2)=\det(t^2 I_n - B^T B)$
\end{answer}
% Thanks!

\begin{example}
Let \(K_{m,n}\) be the complete bipartite graph with biparts of size \(m\) and \(n\) and all possible edges between them, then
\[ A(K_{m,n})=\begin{bmatrix}0 & B \\ B^T & 0\end{bmatrix} \]
where \(B\) is an \(m\times n\) all-one matrix. We have
\[ \underbrace{B^T B}_{n\times n} = m J_n \]
where \(J_n\) is the \(n\times n\) all-one matrix.
Recall that \(J_n\) has exactly one nonzero eigenvalues namely \(n\), so the eigenvalues of \(K_{m,n}\) are \(\sqrt{mn},-\sqrt{mn}\) and 0 (multiplicity \(m+n-2\)).
\end{example}

\begin{example}
Let \(C_{2n}\) be a bipartite graph with adjacency matrix 
\[ A(C_{2n})=\begin{bmatrix}0 & B \\ B^T & 0\end{bmatrix} \]
where \(B\in\R^{n\times n}\) \textit{e.g.}
\[B = \begin{bmatrix}1&0&0&0&1 \\ 1&1&0&0&0 \\ 0&1&1&0&0 \\ 0&0&1&1&0 \\ 0&0&0&1&1 \end{bmatrix}\in\R^{5\times 5} \]
We can verify that
\[ B^T B=2I_n+\underbrace{A(C_n)}_{\begin{bmatrix}0&1&0&0&1 \\ 1&0&1&0&0 \\ 0&1&0&1&0 \\ 0&0&1&0&1 \\ 1&0&0&1&0 \end{bmatrix}} \]
Therefore if \(\lambda_1,\cdots,\lambda_n\) are the eigenvalues of \(C_n\), then the eigenvalues of \(C_{2n}\) are
\[ \pm\sqrt{2+\lambda_k}, \quad k=1,\cdots,n \]
This agrees with our early calculations
 \[ \lambda_k=2\cos\left(\frac{2\pi k}{n} \right) \]
with identity
\[ 1+\cos(2\theta)=2\cos^2\theta \]
\end{example}


\begin{corollary}
If \(G\) is bipartite graph on \(n\) vertices then \(\lambda_n(G)=-\lambda_1(G)\).
\end{corollary}
\begin{remark}
A converse holds for connected graphs.
\end{remark}
\begin{proposition}
If \(G\) is connected on \(n\) vertices and \(\lambda_n(G)=-\lambda_1(G)\), then \(G\) is bipartite.
\end{proposition}
\begin{remark}
It's another consequence of the Perron-Frobenius theorem. We don't prove this.
\end{remark}



\section{Cartesian Products}

\begin{definition}[Cartesian product]
Let \(G\) and \(H\) be graphs, the \emph{Cartesian product} \(G\square H\) has vertex set \(V(G)\times V(H)\) with edges of the forms
\begin{itemize}
\item \((v,w)\sim(v',w)\) where \(v\sim v'\) in \(G\) and \(w\in V(H)\)
\item \((v,w)\sim(v,w')\) where \(w\sim w'\) in \(H\) and \(v\in V(G)\)
\end{itemize}
\end{definition}

\begin{example}
\(P_m\square P_n\) is the \(m\times n\) rectangular lattice.
\(P_2\square P_2\square P_2\) is the 1-skeleton of cube.
In general, \(\underbrace{P_2\square\cdots\square P_2}_{n\text{ copies}}\) is the 1-skeleton of the \(n\)-dimensional hypercube \([0,1]^n\subset\R^{n}\)
\end{example}
\begin{remark}
Cartesian product is associative.
\end{remark}

\begin{definition}[Tensor product]
Let \(\R^m\otimes\R^n\) denote the tensor product of \(\R^m\) and \(\R^n\) which we will identify with the vector space of matrices \(\R^{m\times n}\).
Let \(e^{(i)}\) denote the unit vector (\(i^\text{th}\) entry is one and zeros elsewhere) and define the standard basis of \(\R^{m\times n}\):
\[ \left\{ e^{(i)}e^{(j)T}=\kbordermatrix{ &  &  & j &  & \\  & 0 & 0 & 0 & 0 & 0 \\  & 0 & 0 & 0 & 0 & 0 \\ i & 0 & 0 & 1 & 0 & 0 \\ & 0 & 0 & 0 & 0 & 0 \\ & 0 & 0 & 0 & 0 & 0}, 1\leq i\leq m,1\leq j\leq n \right\} \]
For \(A\in\R^{m\times m},B\in\R^{n\times n}\), let \(A\otimes B\) denote the endomorphism of \(\R^{m\times n}\) given by
\[ (A\otimes B)(M)\coloneqq AMB^T \quad \text{for } M\in\R^{m\times n}. \]
\end{definition}

\begin{lemma}
Let two matrices \(A\in\R^{m\times m},B\in\R^{n\times n}\), then
\[ (A\otimes B)_{(i,j),(k,l)}=A_{i,k}B_{j,l}, \quad (1\leq i,k\leq m,1\leq j,l\leq n) \]
\end{lemma}
\begin{proof}
Recall that the entries of a matrix \(M\in\R^{d\times d}\) are characterized by
\[ Me^{(j)}=\sum_{i=1}^{d}M_{i,j}e^{(i)} \quad \text{for } 1\leq j\leq d. \]
\begin{align*}
(A\otimes B)\left(e^{(k)}e^{(l)T}\right)&=A\left(e^{(k)}e^{(l)T}\right)B^T\\
&=\left(Ae^{(k)}\right)\left(Be^{(l)}\right)^T\\
&=\left(\sum_{i=1}^m A_{i,k}e^{(i)}\right)\left(\sum_{j=1}^n B_{j,l}e^{(j)}\right)^T\\
&=\sum_{i=1}^m \sum_{j=1}^n A_{i,k}B_{j,l} e^{(i)}e^{(j)T}
\end{align*}
\end{proof}

\begin{corollary}
Let \(G\) and \(H\) be graphs with basis of eigenvectors \(x^{(1)},\cdots,x^{(m)}\in\R^m\) and \(y^{(1)},\cdots,y^{(n)}\in\R^n\) corresponding to eigenvalues \(\lambda_1,\cdots,\lambda_m\) and \(\mu_1,\cdots,\mu_n\), then
\[ A(G\square H)=A(G)\otimes I_n + I_m\otimes A(H) \]
and \(G\square H\) has eigenvectors \(x^{(i)}y^{(j)T}\) corresponding to eigenvalues \(\lambda_i+\mu_j\) for \(1\leq i \leq m,1\leq j\leq n\).
\end{corollary}
\begin{remark}
Note that \(\left\{x^{(i)}y^{(j)T}\right\}\) forms a basis of \(\R^{m\times n}\) since they are the unit matrices (matrices with a 1 and 0's elsewhere) under the basis \(x^{(1)},\cdots,x^{(m)}\) of \(\R^m\) and the basis dual to \(y^{(1)},\cdots,y^{(n)}\in\R^n\).
(If \(y^{(1)},\cdots,y^{(n)}\in\R^n\) are orthonormal, it's self-dual).
\end{remark}
\section{Enumerating Walks}

\begin{definition}[Walk]
Let \(G\) be a graph.
A \emph{walk} in \(G\) is a sequence of vertoices \((v_0,\cdots,v_l)\) such that \(v_0\sim v_1\sim v_2\sim\cdots\sim v_l\).
We call \(l\) the \emph{length} of the walk.
(A \emph{path} is a walk with no vertices repeated except possibly \(v_0=v_l\).)
We call a walk \emph{closed} if \(v_0=v_l\)
(a closed path is called cycle).
We will consider \emph{marked closed walks}, \textit{i.e.}, closed walks whose starting vertex is distinguished (and an occurance of that vertex within the walk if it appears multiple times).
\end{definition}

\begin{example}
Triangle has 6 marked closed walks:
\begin{center}
\begin{tabular}{cc}
1 2 3 & 1 3 2\\
2 1 3 & 2 3 1\\
3 1 2 & 3 2 1
\end{tabular}
\end{center}
\end{example}

\begin{lemma}
Let \(G\) be a graph and \(l\in\N\).
For any \(v,w\in V(G)\), the number of walks of length \(l\) from \(v\) to \(w\) in \(G\) equals
\[ \left(A(G)^l\right)_{v,w}, \quad (A(G)^0=I ) \]
\end{lemma}
\begin{proof}
\begin{align*}
    \left(A(G)^l\right)_{v,w}&=\sum_{u_1,\cdots,u_{l-1}\in V(G)} A(G)_{v,u_1} A(G)_{u_1,u_2}\cdots A(G)_{u_{l-2},u_{l-1}} A(G)_{u_{l-1},w}\\
&=\sum_{u_1,\cdots,u_{l-1}\in V(G)} 1_{v\sim u_1\sim u_2\sim\cdots\sim u_{l-1}\sim w } 
\end{align*}
\end{proof}

\begin{corollary}
Let \(G\) be a graph on \(n\) vertices with adjacency matrix \(A\) and \(l\in\N\).
\begin{enumerate}[(i)]
\item the number of marked closed walks in \(G\) of length \(l\) equals
\[ \sum_{v\in V(G)} \left(A^l\right)_{v,v}=\mathrm{tr}\left(A^l\right)=\sum_{j=1}^{n}\lambda_j^l(G) \]
\item the total number of walks in \(G\) of length \(l\) equals
\[ \sum_{v,w\in V(G)} \left(A^{l}\right)_{v,w}=\text{sum of entries of }A^l \]
\end{enumerate}
\end{corollary}

\begin{example}
Let \(G=K_n\) the complete graph on \(n\) vertices.
Recall that the eigenvalues of \(G\) are \(n-1,-1,-1,\cdots,-1\).\
Hence, the number of marked closed walks of length \(l\) is
\[ (n-1)^{l}+(n-1)(-1)^l \]
Now let's find the total number of walks of length \(l\).
Recall \(A(G)=J_n-I_n\).
Therefore by binomial theorem
\[ (J_n-I_n)^l=\sum_{d=0}^{l}\binom{l}{d}(-1)^{l-d}J_n^d \]
By \(J_n^2=nJ_n\) anmd induction,
\[ J_n^d=n^{d-1}J_n, \quad \forall d\geq1 \]
So the sum of the entries of \(J_n^d\) is \(n^{d-1}n^2=n^{d+1}\) for all \(d\in\N\).
Then the sum of entries of \(A(G)^l\), (\textit{i.e.}, number of walks of length \(l\)) is
\[ \sum_{d=0}^l\binom{l}{d} (-1)^{l-d}{n}^{d+1}=n(n-1)^l \]
Finally, note that by symmetry \(\left(A(G)^l\right)_{v,w}\) is the same for all distinct \(v,w\in V(G)\).
\begin{align*}
\left(A(G)^l\right)_{v,w}&=\frac{\text{sum of entries of }A(G)^l-\text{sum of diagonal entries of }A(G)^l}{\# \text{ of off-diagonal entries}}\\
&=\frac{n(n-1)^l-\left[(n-1)^l+(n-1)(-1)^l\right]}{n^2-n}
=\frac{(n-1)^l-(-1)^l}{n}
\end{align*}
\end{example}

\begin{corollary}
The number of triangles in a graph \(G\) is
\[ \frac{1}{6}A(G)^3 \]
The coefficient \(\frac{1}{6}\) comes from the fact that a triangle has 6 walks of length 3.
\end{corollary}

\begin{lemma}
Let \(M\in\R^{n\times n}\). The sum of the entries of \(M\) is
\[ \bar{e}^T M \bar{e}=\mathrm{tr}(MJ_n)=\det(I_n+MJ_n)-1 \]
where \(\bar e=\begin{bmatrix} 1 & 1 & \cdots & 1 \end{bmatrix}^T\in\R^n \).
\end{lemma}
\begin{proof}
We can check explicitly that \(\bar{e}^TM\bar{e}\) is the sum of entries.
\begin{enumerate}[(i)]
\item \[ \underbrace{\bar{e}^TM\bar{e}}_{1\times1}=\mathrm{tr}\left(\bar{e}^TM\bar{e} \right)=\mathrm{tr}\left(M\bar{e}\bar{e}^T \right)=\mathrm{tr}(MJ_n) \]
\item Note that \(\mathrm{rank}(MJ_n)\leq1\).
For any matrix \(N\in\R^{n\times n}\) of rank at most 1, we have the \emph{Sherman-Morrison identity}
\[ \det(I_n+N)=1+\mathrm{tr}(N) \]
This is true since \(N\) has at most one nonzero eigenvalue \(\lambda\) and both sides equal to \(1+\lambda\).
\end{enumerate}
\end{proof}

\section{Walk Generating Functions}
\begin{definition}[ordinary generating function]
The \emph{ordinary generating function} (ogf) of the sequence \((a_l)_{l\in\N}\) is the formal power series
\[ \sum_{l=0}^\infty a_l t^l \]
It is called ordinary to distinguish from the \emph{exponential generating function}
\[ \sum_{l=0}^\infty \frac{a_l}{l!} t^l \]
\end{definition}
Recall the geometric series formula
\[ \sum_{l=0}^\infty t^l=\frac{1}{1-t} \]
We can think of \(\frac{1}{1-t}\) as a rational function of \(t\), or as the inverse of the formal power series \(1-t\).
More generally,
\[ \sum_{l=0}^\infty (tA)^l=(I_n-tA)^{-1} \]


\begin{definition}
Let \(G\) be a graph of \(n\) vertices with adjacency matrix \(A\).
\begin{enumerate}[(i)]
\item For \(v,w\in V(G)\), let \(W_{v,w}^G(t)\) denote the ordinary generating function for number of walks from \(v\) to \(w\) with respect to lenth.
\[ W_{v,w}^G(t)=\sum_{l=0}^\infty (A^l)_{v,w}t^l=\left((I_n-tA)^{-1}\right)_{v,w} \]
\item Let \(W_{\text{closed}}^G(t)\) denote the ordinary generating function for marked closed walks in \(G\) with respect to lenth.
\[ W_{\text{closed}}^G(t)=\sum_{l=0}^\infty \mathrm{tr}(A^l)t^l=\mathrm{tr}\left((I_n-tA)^{-1}\right) \]
\item Let \(W_{\text{all}}^G(t)\) denote the ordinary generating function for all walks in \(G\) with respect to lenth.
\[ W_{\text{all}}^G(t)=\sum_{l=0}^\infty \underbrace{\mathrm{tr}(A^lJ_n)}_{\text{sum of entries of }A^l}t^l=\mathrm{tr}\left((I_n-tA)^{-1}J_n\right) \]
\end{enumerate}
\end{definition}
\begin{example}
Recall 
\[ \tr\left({A(K_n)^l}\right)=(n-1)^l+(n-1)(-1)^l \]
Therefore,
\begin{align*}
W_\text{closed}^{K_n}(t)&=\sum_{l=0}^\infty\left[(n-1)^l+(-1)^l(n-1)\right]t^l\\
&=\frac{1}{1-(n-1)t}+(n-1)\frac{1}{1-t}
\end{align*}
\end{example}

\begin{theorem}
Let \(G\) be a graph with \(n\) vertices.
\[ W_\text{closed}^G(t)=\tr\left((I_n-tA(G))^{-1}\right)=\sum_{j=1}^n\frac{1}{1-t\lambda_j(G)}=\frac{\phi'_G\left(\frac{1}{t}\right)}{t\phi_G\left(\frac{1}{t}\right)} \]
\end{theorem}
\begin{proof}
The eigenvalues of \(A(G)\) are \(\lambda_1,\lambda_2,\cdots,\lambda_n\), so the eigenvalues of \((I_n-tA(G))^{-1}\) are
\[ (1-t\lambda_1(G))^{-1}, \cdots, (1-t\lambda_n(G))^{-1}\]
Thus,
\[ \tr\left((I_n-tA(G))^{-1}\right)=(1-t\lambda_1(G))^{-1}+\cdots+(1-t\lambda_n(G))^{-1} \]
On the other hand,
\begin{align*}
\frac{\phi'_G\left(\frac{1}{t}\right)}{t\phi_G\left(\frac{1}{t}\right)}&=
\frac{1}{t}\left.\frac{\mathrm{d}}{\mathrm{d} s}\right|_{s=\frac{1}{t}}\log \phi_G(s)\\
&=\frac{1}{t}\left.\frac{\mathrm{d}}{\mathrm{d} s}\right|_{s=\frac{1}{t}}\log[(s-\lambda_1(G))\cdots(s-\lambda_n(G))]\\
&=\frac{1}{t}\left.\left(\frac{1}{s-\lambda_1(G)}+\cdots+\frac{1}{s-\lambda_n(G)} \right)\right|_{s=\frac{1}{t}}\\
&=\frac{1}{1-t\lambda_1(G)}+\cdots+\frac{1}{1-t\lambda_n(G)}
\end{align*}
Alternative proof using the identity
\[ \tr(\log(A))=\log(\det(A)) \]
as functions on matrices.
\end{proof}


\begin{example}
Let \(G=K_n\), then
\begin{align*}
\phi_{K_n}(t)&=(t-(n-1))(t+1)^{n-1}\\
\phi'_{K_n}(t)&=(t+1)^{n-2}(tn-n^2+2n)
\end{align*}
Then
\begin{align*}
W_\text{closed}^{K_n}(t)=\frac{\left(1+\frac{1}{t}\right)^{n-2}\left(\frac{n}{t} -n^2+2n\right)}{t\left(\frac{1}{t}-(n-1)\right)\left(\frac{1}{t}+1\right)}
\end{align*}
\end{example}


\section{Asymptotic Behavior}
\begin{theorem}
Let \(G\) be a connected non-bipartite graph with largest eigenvalue \(\lambda_1\) corresponding to an eigenvector \(x\) with \(||x||=1\).
(recall from the Perron-Frobenius theorem that \(|\lambda_j(G)|<\lambda_1\) for all \(j>1\)), then
\[ \lim_{l\to\infty}\frac{A(G)^l}{\lambda_1^l}=xx^T \]
In particular, for all \(v,w\in G\)
\[ A(G)_{v,w}^l\sim \lambda_1^l x_v x_w \]
\end{theorem}
\begin{proof}
Let's assume that \(V(G)=[n]\).
Let \(x^{(1)},x^{(2)},\cdots,x^{(n)}\in\R^n\) be an orthonormal basis of eigenvectors of \(A(G)\) corresponding to eigenvalues \(\lambda,\lambda_2,\cdots,\lambda_n\).
Recall that
\[ A(G)=\sum_{j=1}^{n}\lambda_j x^{(j)} x^{(j)T} \]
Since \(x^{(i)T}x^{(j)}=\delta_{ij}\), we get
\[ A(G)^l=\sum_{j=1}^{n}\lambda_j^l x^{(j)} x^{(j)T} \]
Therefore,
\[ \frac{A(G)^l}{\lambda_1^l}=xx^T+\sum_{j=2}^{n}\left(\frac{\lambda_j}{\lambda_1}\right)^lx^{(j)} x^{(j)T} \]
Since \(|\lambda_j|<\lambda_1\) for \(j>1\), we get
\[ \lim_{l\to\infty}\frac{A(G)^l}{\lambda_1^l}=xx^T \]
\end{proof}
\begin{remark}
\begin{enumerate}
\item Connected: \(\lambda_1\) has multiplicity 1.
\item Non-bipartite: \(-\lambda_1\) not an eigenvalue.
\end{enumerate}
For connected bipartite graphs, we have the following
\[ A(G)^l_{v,w}=\begin{cases}0 & l\not\equiv d(v,w) \mod 2 \\ \sim 2\lambda_1^l x_v x_w & l\equiv d(v,w) \mod 2  \end{cases} \]
\end{remark}

\begin{example}
Consider \(G=P_n\), which is connected and bipartite.
Recall the eigenvalues of \(G\) are
\[ \lambda_k=2\cos\left(\frac{\pi k}{n+1}\right) \quad k=1,\cdots,n \]
We can check that an eigenvector \(x\in\R^n\) with \(||x||=1\) corresponding to \(\lambda_1=2\cos\left(\frac{\pi}{n+1}\right)\) is given by
\[ x_j=\sqrt{\frac{2}{n+1}}\sin\left(\frac{\pi j}{n+1}\right) \]
(entries bigger if closer to the middle of the path).
An eigenvector \(y\in\R^n\) with \(||y||=1\) corresponding to \(-\lambda_1\) is given by
\[ y_j=(-1)^{j-1}x_j \quad j=1,\cdots,n \]
Hence, as \(l\to\infty\), we have
\[ \left(A(G)^l\right)_{ij}=\begin{cases}0 & l\not\equiv |i-j| \mod 2 \\ \sim 2\left(2\cos\left(\frac{\pi}{n+1}\right)^l \right)\left(\frac{2}{n+1}\right)\sin\left(\frac{\pi i}{n+1}\right)\sin\left(\frac{\pi j}{n+1}\right) & l\equiv |i-j| \mod 2  \end{cases} \]
Why does \(||x||=1\)?
\[ \sum_{j=1}^{n}\sin^2\left(\frac{\pi j}{n+1}\right)=\frac{1}{2}\sum_{j=0}^{2n+1}\sin^2\left(\frac{\pi j}{n+1}\right) \]
Recall
\[ \sin\theta=\frac{e^{i\theta}-e^{-i\theta}}{2i} \]
Letting \(\rho=\frac{2\pi i}{2n+2}\),
\[ \sin\left(\frac{j\pi}{n+1}\right)=\frac{\rho^j-\rho^{-j}}{2i} \]
Thus,
\begin{align*}
\frac{1}{2}\sum_{j=0}^{2n+1}\sin^2\left(\frac{\pi j}{n+1}\right)&=
\frac{1}{2}\sum_{j=0}^{2n+1}\left(\frac{\rho^j-\rho^{-j}}{2i}\right)\\
&=\frac{1}{2}\sum_{j=0}^{2n+1}\left(\frac{\rho^{2j}-2+\rho^{-2j}}{-4}\right)\\
&=-\frac{1}{8}\left(\underbrace{\sum_{j=0}^{2n+1}\rho^{2j}}_{=0}+\underbrace{\sum_{j=0}^{2n+1}-2}_{=-2(2n+2)}+\underbrace{\sum_{j=0}^{2n+1}\rho^{-2j}}_{=0} \right)\\
&=\frac{n+1}{2}
\end{align*}
\end{example}
\section{Graph Homomorphisms}
\begin{definition}
Let $G$ and $H$ be graphs. We say that $\phi:V(G)\to V(H)$ is a homomorphism if
\[ \forall v,w\in V(G), \quad v\sim_{G} w\implies \phi(v)\sim_{H} \phi(w). \]
We call $\phi$ an isomorphism if $\phi$ is a bijection and its inverse $\phi^{-1}$ is also a homomorphism.
If \(\phi\) is a bijection, it is an isomorphism \iff
\[ \forall v,w\in V(G),\quad v\sim_{G} w \Longleftrightarrow \phi(v)\sim_{H} \phi(w) \]
\end{definition}
\begin{definition}[Automorphism]
An automorphism of $G$ is an isomorphism from $G$ to $G$.

\textit{Note: $(Aut(G), \circ)$ forms a group.}
\end{definition}

Let $\mathfrak{S}(V)$ denote the symmetric group of all permutations (i.e. bijections) of the set $V$ (We also let $\mathfrak{S}_n$ denote the symmetric group of permutations of $[n]=\{ 1,2,...,n \}$). $Aut(G)$ is a subgroup of $\mathfrak{S}(V(G))$.

\begin{example}
$\mathrm{Aut(K_n)} = \mathfrak{S}_n$,
$\mathrm{Aut(C_n)} = D_n$(the dihedral group of order $2n$ of rotations and reflections of a regular $n$-gon).
\end{example}

\begin{theorem}[Lie Theory]
Miss this part...
\end{theorem}

\begin{example}
$n\geq 2$, $P_n$ has 2 automorphism (the identity map and the map that reflects the path about its center).
\end{example}

\begin{definition}[\(m\)-coloring]
\(m\)-coloring of a graph $G$ is a labeling of its vertices by a color $\varphi(V) \in [m]$.
An \(m\)-coloring is called \textbf{proper} if no two adjacent vertices are given the same color, i.e.,\(\varphi \text{ is proper }\) \iff \(\varphi\text{ is homomorphism form }G \text{ to }K_m\).
\end{definition}

\begin{definition}
A permutation matrix is a square $\{0,1\}$ matrix with exactly one $1$ in every row and column. For each permutation $\pi\in\mathfrak{S}_n$. Let $P(\pi)$ denote the $n\times n$ permutation matrix where $j^{\textrm{th}}$ column has a $1$ in row $\pi(j)$ for all $j\in[n]$.
\end{definition}
\begin{example}
Let $\pi=\begin{pmatrix}
1&2&3\\
2&3&1
\end{pmatrix}$, $\sigma=\begin{pmatrix}
1&2&3\\
2&1&3
\end{pmatrix}$. Then 
\[P(\pi)=\begin{pmatrix}
0&0&1\\
1&0&0\\
0&1&0
\end{pmatrix},\qquad 
P(\sigma)=\begin{pmatrix}
0&1&0\\
1&0&0\\
0&0&1
\end{pmatrix}.\]
Note that $P(\pi)P(\sigma)=P(\pi\sigma).$ Let $P_n$ be the set of all $n\times n$ permutation matrices. Then it's easy to verify that $P_n\leq \mathrm{GL}_n(\mathbb{R})$ and $P_n\cong \mathfrak{S}_n$.
\end{example}

\begin{definition}
Define the sign or signature $\mathrm{sgn}(\pi)$ of $\pi\in\mathfrak{S}_n$ by $\mathrm{sgn}(\pi)=\det(P(\pi))$.
\end{definition}

\begin{lemma}
Let $\pi\in\mathfrak{S}_n$, then
\begin{enumerate}[(i)]
\item If $\pi$ can be represented as a product of $m$ transpositions, then $\mathrm{sgn}(\pi)=(-1)^m$.
\item $\mathrm{sgn}(\pi) = (-1)^{\#\, i<j:\pi(j)>\pi(i)}$.
\end{enumerate}
\end{lemma}

\begin{proof}
\begin{enumerate}[(i)]
\item Using multiplicativity of the determinant, it suffices to prove that every transposition has sign $-1$. This is true since the permutation matrix of a transposition is
\[
\begin{pmatrix}
0&1&0&\cdots&0\\
1&0&0&\cdots&0\\
0&0&1&\cdots&0\\
\vdots &\vdots&\vdots&\ddots&\vdots\\
0&0&0&\cdots&1
\end{pmatrix}
,\]
up to a permutation of the basis vectors.
\item Check that 
\begin{enumerate}
\item $\mathrm{sgn}(\mathrm{id})=1$.
\item Since every transposition has sign -1, note that if there is inversion ($(i,j): i < j, \pi(j) > \pi(i)$) in a permutation, then there must exist an inversion between 2 consecutive indices (e.g. $\pi = 3124$ contains inversion (1,2) and (1,3)). Now it is enough to note that one could reverse such an inversion using transposition. So the total number of transpositions needed is the number of inversions.
\end{enumerate}
\end{enumerate}
\end{proof}

\begin{example}
$\mathrm{sgn}\begin{pmatrix}
1&2&3\\
2&3&1
\end{pmatrix}=1$
\end{example}

\begin{proposition}
Let $G$ and $H$ be graphs on the vertex set $[n]$ and $\pi\in\mathfrak{S}_n$. Then $\pi$ is an isomorphism from $G$ to $H$ if and only if 
 \[A(G)=P(\pi)^{-1}A(H)P(\pi).\]
\end{proposition}
\begin{proof}
Note that $(P(\pi)^{-1}A(H)P(\pi))_{i,j}=A(H)_{\pi(i),\pi(j)}.$ Therefore we have an equivalence chain,
\begin{align*}
&A(G)=P(\pi^{-1})A(H)P(\pi)\\
\Longleftrightarrow\quad & A(G)_{i,j}=A(H)_{\pi(i),\pi(j)}\\
\Longleftrightarrow\quad & (i\sim j\,\Longleftrightarrow\,\pi(i)\sim\pi(j))\\
\Longleftrightarrow\quad &\pi \textrm{ is an isomorphism}
\end{align*}
\end{proof}

\begin{corollary}
The automorphisms of a graph $G$ correspond to permutation matrices $P$ satisfying $A(G)=P^{-1}A(G)P$.
\end{corollary}

\begin{theorem}
Suppose that the eigenvalues of $G$ are all distinct. Then every automorphism of $G$ has order at most $2$.
\end{theorem}

\begin{proof}
Let $A=A(G)\in\mathbb{R}^{n\times n}$. Every automorphism of $G$ corresponds to a permutation matrix $P$ with $AP=PA$. We must show that $P^2=I_n$.

Let $\lambda_1,\ldots,\lambda_n\in\mathbb{R}$ be the eigenvalues of $A$ with eigenbasis $x^{(1)},\ldots,x^{(n)}\in\mathbb{R}^n$. Then for $i\in[n]$, we have 
\begin{align*}
&AP=PA\implies APx^{(i)}=PAx^{(i)}\implies A(Px^{(i)})=\lambda_i Px^{(i)}
\implies  Px^{(i)}=\mu_i x^{(i)}
\end{align*}
Since $P^{n!}=I_n$, $(\mu_i)^{n!}=1$. Thus $\mu_i=\pm 1$. We get $P^2x^{(i)}=\mu_i^2x^{(i)}=x^{(i)}$ for all $i\in[n]$. Since $\{x^{(i)}\}$ forms a basis of $\mathbb{R^n}$, $P^2$ is $I_n$.
\end{proof}
\begin{remark}
in fact, almost all graphs have a trivial automorphism group, i.e.
\[\lim_{n\to\infty}\frac{\#\textrm{graphs $G$ with $V(G)=[n]$ and $\left|\mathrm{Aut}(G)\right|=1$}}{\#\textrm{graphs $G$ with $V(G)=[n]$}}=1.\]
See section 2.3 of Godsil-Royle for a proof.
\end{remark}
\begin{remark}[Graph Isomorphic Problem]
\begin{flushleft}\end{flushleft}
\begin{itemize}
\item Babai \& Luks(1983)  $2^{O(\sqrt{n\log n})}$
\item Babai(2017) $2^{O((\log n)^c)}$
\end{itemize}
\end{remark}
% Friday 1.19
\section{Expander Graphs}
Vaguely speaking, an expander graph is a sparse graph (\textit{i.e.}, it has few edges) which is nonetheless highly connected.
Expander graphs have applications to
\begin{itemize}
\item constructing error-correcting codes
\item derandomize algorithms
\item serve methods in number theory
\item hyperbolic manifolds
\end{itemize}

\begin{definition}[edge expansion ratio]
Lety \(G\) be a graph.
For \(S\subseteq V(G)\) denote the set of edges with one endpoint in \(S\) and one endpoint not in \(S\) by \(\partial S\).
Define the edge expansion ratio
\[ h(G)\coloneqq \min_{\substack{S\subseteq V(G)\\|S|\leq|V(G)|/2}} \frac{|\partial S|}{|S|} \]
\end{definition}
\begin{definition}
A sequence of \(k\)-regular graphs \((G_n)_{n\in\N}\) is called a family of expander graphs if
\[ |V(G_0)|<|V(G_1)|<\cdots \]
and there exists \(\varepsilon>0\) such that \(h(G_n)\geq0,\forall n\in\N\).
\end{definition}
\begin{remark}
It's relatively easy to show that such families exist by probabilistic arguments, but it's relatively difficult to construct them explicitly.
\end{remark}
\begin{theorem}
Suppose \(G\) is connected and \(k\)-regular so \(\lambda_1(G)=k\), then
\[ \frac{k-\lambda_2(G)}{2}\leq h(G)\leq \sqrt{2k(k-\lambda_2(G))} \]
\end{theorem}
Therefore a family \((G_n)_{n\in\N}\) of \(k\)-regular graphs with \(|V(G_0)|<|V(G_1)|<\cdots\) forms a family of expander graphs \iff there exists \(\varepsilon'>0\) such that
\[ k=\lambda_2(G_n)\geq\varepsilon' \quad \forall n\in\N \]
The spectral gap \(k-\lambda_2(G)\) or sometimes \( \min\{|k-\lambda_2{G}|,k-|\lambda_n(G)|\} \) measures how rapidly a random walk on \(G\) mixes.



\chapter[Tilings \& Spanning Trees]{Tilings, Spanning Trees and Electrical Networks}

\begin{itemize}
\item motivation (and some results) come from Physics, Chemistry and Computer Science.
\end{itemize}

\section{Tilings and Perfect Matchings}
We'll prove the following theorems.
\begin{theorem}[Kasteleyn (1961)]
Let \(m,n\in\N\) be such that \(mn\) is even.
The number of domino tilings of an \(m\times n\) board equals
\[ T(m,n)=\prod_{j=1}^{m}\prod_{k=1}^n \left(4\cos^2\left(\frac{\pi j}{m+1}\right) + 4\cos^2\left(\frac{\pi k}{n+1}\right)\right)^{\frac{1}{4}} \]
\end{theorem}
\begin{example}
\(\pi(2,3)\) has 3 tilings.
\begin{figure}[h]
    % Add graph
    \caption{Three tilings of \(\pi(2,3)\)}
\end{figure}
\end{example}
\begin{remark}
If \(m,n\) both odd, there's no domino tilings of an \(m\times n\) board.
\end{remark}


\begin{definition}[Perfect matching]
A \emph{perfect matching} of a graph \(G\) is a subset of its edges which meets every vertex exactly once.
\end{definition}
\begin{remark}
\begin{flushleft}\end{flushleft}
\begin{itemize}
    \item Only graphs with even number of vertices can have a perfect matching.
    \item Note that domino tiling of an \(m\times n\) board correspond to perfect matchings of the \(m\times n\) grid \(P_m\square P_n\).
\end{itemize}
\end{remark}
\begin{example}
\(K_4\) has 3 perfect matchings. Each of them are rotated from anther.
\begin{figure}[ht]
    % Add graph
    \caption{Three perfect matchings of \(K_4\)}
\end{figure}
\end{example}
We will present a general method for conducting perfect matchings of certain graphs due to Kasteleyn, called Pfaffian method.

\section{Skew-Symmetric Matrices}
\begin{definition}[skew-symmetric matrix]
A matrix \(A\in\R^{n\times n}\) is called \emph{skew symmetric} if
\[ A^T=-A. \]
\end{definition}
\begin{remark}
Note that for skew-symmetric \(A\),
\[ \det(A)=\det(A^T)=\det(-A)=(-1)^n\det(A) \]
Therefore if \(n\) is odd, \(\det(A)=0\).
However, if \(n\) is even, \(\det(A)\) is a perfect square in the entries of \(A\).
\end{remark}
\begin{example}
\[ \det\begin{bmatrix}0 & a & b & c \\ -a & 0 & d & e \\ -b & -d & 0 & f \\ -c & -e & -f & 0 \end{bmatrix}=(af-be+cd)^2=\text{Pfaffian}^2 \]
\end{example}
\begin{definition}[Pfaffian]
Let \(A\in\R^{2n\times 2n}\) be skew-symmetric. We define the \emph{Pfaffian} of \(A\)
\[ \mathrm{pf}(A)\coloneqq \sum_{\substack{\text{perfect matching of }K_{2n}\\ M=\{\{i_1,j_1\},\cdots,\{i_n,j_n\}\} }} \mathrm{sgn}\begin{pmatrix} 1&2&3&4&\cdots&2n-1&2n\\ i_1&j_1&i_2&j_2&\cdots&i_n&j_n\end{pmatrix}\prod_{r=1}^n A_{i_r,j_r} \]
\end{definition}
\begin{example}
Let \(A\in\R^{4\times4}\) be skew-symmetric, then
\begin{align*}
    \mathrm{pf}(A)&=\mathrm{sgn}\begin{pmatrix} 1&2&3&4\\ 1&2&3&4\end{pmatrix}A_{1,2}A_{3,4}\\
    &+\mathrm{sgn}\begin{pmatrix} 1&2&3&4\\ 1&3&2&4\end{pmatrix}A_{1,3}A_{2,4}\\
    &+\mathrm{sgn}\begin{pmatrix} 1&2&3&4\\ 1&4&2&3\end{pmatrix}A_{1,4}A_{2,3}\\
    &=A_{1,2}A_{3,4}-A_{1,3}A_{2,4}+A_{1,4}A_{2,3}
\end{align*}
\end{example}
\begin{remark}
Why is the Pfaffian well-defined?
\begin{enumerate}[(a)]
    \item the M-term does not change if we swap the order of a given edge
    \[ \{i_r,j_r\} \longleftrightarrow \{j_r,i_r\} \]
    \begin{itemize}
        \item \(\mathrm{sgn}\begin{pmatrix} \cdots&2r-1&2r&\cdots \\ \cdots&i_r&j_r&\cdots \end{pmatrix}=-\mathrm{sgn}\begin{pmatrix} \cdots&2r-1&2r&\cdots \\ \cdots&j_r&i_r&\cdots \end{pmatrix} \)
        \item \(A_{i_r,j_r}=-A_{j_r,i_r}\)
    \end{itemize}
    \item the M-term does not change if we swap two consequtive edges
    \[ \{i_r,j_r\} \longleftrightarrow \{i_{r+1},j_{r+1}\} \]
    \begin{itemize}
    \item \(\mathrm{sgn}\begin{pmatrix} \cdots&2r-1&2r&2r+1&2r+2&\cdots \\ \cdots&i_r&j_r&i_{r+1}&j_{r+1}&\cdots \end{pmatrix} \)
    \item unchanged
    \begin{question}
        This part is not clear enough.
    \end{question}
\end{itemize}
\end{enumerate}
So we can compute
\[ \mathrm{pf}(A)=\underbrace{\frac{1}{2^nn!}}_{\substack{\text{there are }\frac{(2n)!}{2^nn!}\\ \text{ perfect matchings}}}\sum_{\pi\in\mathfrak{S}_{2n}} \mathrm{sgn}(\pi)\prod_{i=1}^{2n}A_{\pi(2i-1),\pi(2i)} \]
\end{remark}
% Monday 1.22
\section[Pfaffian method]{Pfaffian method for connecting perfect matchings}

% Wednesday 1.24
\begin{theorem}
Every planar graph has a Pfaffian orientation.
\end{theorem}
\begin{theorem}[EDIT: Kuratowski]
A graph is planar if and only if there is no subdivision isomorphic to $K_{3,3}$ or $K_5$.
\end{theorem}
\begin{theorem}
A graph has a Pfaffian orientation if and only if no $K_{3,3}$.
\end{theorem}
\begin{theorem}
The number of pilings using dominos on an $m\times n$ board is 
$$T(m,n)=\prod_{j=1}^{m}\prod_{k=1}^n(4\cos^2(\frac{\pi j}{m+1})+4\cos^2(\frac{\pi k}{n+1}))^{\frac{1}{4}}.$$
\end{theorem}
\begin{proof}
Any two domino tilings are related by a sequence of flips.
\begin{figure}[ht!]
\centering
\begin{tikzpicture}[baseline=(current bounding box.center)]
    \pgfmathsetmacro{\r}{0.60};
    \foreach \x in {0,...,2}{
    \foreach \y in {0,...,2}{
    \node[inner sep=0](v\x\y)at(\r*\x,\r*\y){};}}
    \draw[thick](v00.center)--(v20.center)--(v22.center)--(v02.center)--cycle;
    \path[thick](v10.center)edge(v12.center);
    \draw[thick,->] (1.5,0.6) -- (3.7,0.6);
  \begin{scope}[xshift=4cm]
    \foreach \x in {0,...,2}{
    \foreach \y in {0,...,2}{
    \node[inner sep=0](v\x\y)at(\r*\x,\r*\y){};}}
    \draw[thick](v00.center)--(v20.center)--(v22.center)--(v02.center)--cycle;
    \path[thick](v01.center)edge(v21.center);
  \end{scope}
\end{tikzpicture}
\label{fig:domino_flip}
\caption{Flip}
\end{figure}
Any two perfect matchings contribute to the same sgn. Thus the number of perfect matchings of $P_m\square P_n$ is $|Pf(\vec{A}(D))|=\sqrt{|\det(\vec{A}(G))|}$. By rescaling you get $|\det(A)|=|\det(A'')|$, where 
$$A''=I_m\otimes A(P_n)-i(A(P_m)\otimes I_n).$$
Recall that the eigenvalues of $P_n$ are $2\cos(\frac{\pi k}{n+1}),k\in[n]$. Therefore, eigenvalues of $A''$ are 
$$2\cos(\frac{\pi k}{n+1})-i2\cos(\frac{\pi j}{m+1}).$$
Then we have the number of perfect matchings 
\begin{align*}
&|Pf(\vec{A}(D))|=\sqrt{|\det(A)|}\\
=&|\det(A'')|^{\frac{1}{2}}\\
=&\prod_{j=1}^{m}\prod_{k=1}^n |4\cos^2(\dfrac{\pi k}{n+1}) +4\cos^2(\dfrac{\pi k}{n+1})|^{1/4}
\end{align*}


\end{proof}

\subsection{Asymptotic of \texorpdfstring{$T(m,n)$}{T(m,n)}}
We have
\begin{align*}
\dfrac{\log (T(m,n))}{mn} &= \dfrac{1}{mn} \log \prod_{j=1}^{m}\prod_{k=1}^n(4\cos^2(\frac{\pi j}{m+1})+4\cos^2(\frac{\pi k}{n+1}))^{\frac{1}{4}}\\
&=\dfrac{1}{4\pi^2}(\dfrac{\pi}{m})(\dfrac{\pi}{n})\sum\limits_{j=1}^m\sum_{k=1}^n\log (4\cos^2(\dfrac{\pi j}{m+1})+4\cos^2(\dfrac{\pi k}{n+1}))\\
\xrightarrow{m,n \rightarrow \infty}&=\dfrac{1}{4\pi^2}\int_0^{\pi}\int_0^{\pi}\log(4\cos ^2x+4\cos^2y)\mathrm{d}x \mathrm{d}y\\
&=\dfrac{G}{\pi}
\end{align*}
where $G = \dfrac{1}{4\pi}\int_0^{\pi}\int_0^{\pi}\log(4\cos ^2x+4\cos^2y)\mathrm{d}x \mathrm{d}y=1-\dfrac{1}{3^2}+\dfrac{1}{5^2}+...$

(Open Problem) is G irrational?

$\log T(m,n) \approx G_{m,n}$. So $T(m,n) \approx \mathrm{e}^{G_{m,n}/\pi}$



\begin{theorem}
$$N(a,b,c) = \prod_{i=1}^a\prod_{j=1}^b\prod_{k=1}^c \dfrac{i+j+k-1}{i+j+k-2}$$
\end{theorem}
\begin{proof}
We will apply Jacobi's Identity:
$$\det(A)\det(A_{[2,n-1][2,n-1]})=\det(A_{[2,n][2,n]})\det(A_{[1,n-1][1,n-1]}) - \det(A_{[2,n][1,n-1]})\det(A_{[1,n-1][2,n]})$$
for any $n\times n$ matrix $A$. This is a special case of general determinant identities called Grassmann-pliicker relations (or Schouter(???) identities in physics). In our case, we have
$$N(a,b,c)N(a,b,c-2) = N(a,b,c-1)N(a,b,c-1)-N(a-1,b+1,c-1)N(a+1,b-1,c-1),$$
where $N(a,b,0)=1, N(a,b,c) = {a+b\choose a}$. 
Now just need to check that
$$\prod_{i=1}^a\prod_{j=1}^b\prod_{k=1}^c \dfrac{i+j+k-1}{i+j+k-2}$$
satisfies the same recursion.
\end{proof}
There is a direct way to relate determinants to enumerating perfect matchings.

\section{Aztee Diamond}
The $n^{\textrm{th}}$ Aztee diamond is a symmetric board with rows of length $2,4,6,\ldots,2n,2n,\ldots,6,4,2$.
\begin{theorem}
The $n^\textrm{th}$ Aztee diamond has $2^{n+1\choose 2}$ domino tilings.
\end{theorem}

\section{Spanning Trees}
\begin{theorem}
Let $m,n\in\mathbb{N}$ be odd. The number of domino tilings of an $m\times n$ board with one corner square removed is 
$$\prod_{j=1}^{\floor{\frac{m}{2}}}\prod_{k=1}^{\floor{\frac{n}{2}}}(4\cos^2(\frac{\pi j}{m+1})+4\cos^2(\frac{\pi k}{n+1}))$$
\end{theorem}
To prove this theorem we first construct a bijection from domino tilings to spanning trees of a grid graph. Then we will count spanning trees using the matrix-tree theorem.
\begin{definition}
A spanning subgraph $H$ of $G$ is a graph $V(H)=V(G)$ and $E(H)\subset E(G)$. If $H$ is a tree, we call $H$ a spanning tree of $G$.
\end{definition}
\begin{theorem}
let $m,n\in\mathbb{Z}_{>0}$, define a map $\varphi$ from the set of domino tilings of a $(2m-1)(2n-1)$ board whose north east corner is removed to the set of spanning trees of $P_m\square P_n$ as follows: 

Embed $P_m\square P_n$ inside the (2m-1)(2n-1) board. Given a domino tiling $M$, let $\varphi(M)$ be the spanning tree whose edges cross only one edge of $M$. This $\varphi$ is a bijection.
\end{theorem}
\begin{proof}
First we prove $\varphi$ is well-defined. This can be seen from the fact that 
\begin{enumerate}
\item the edges of $\varphi(M)$ correspond to  the vertices of $P_m\square P_n$ except for the northeast corner vertex. So $\varphi(M)$ has $mn-1$ edges,
\item $\varphi(M)$ is acyclic, otherwise, the path would enclose odd number of squares, which cannot be filled with dominoes.
\end{enumerate}

Then we must construct the unique domino tiling $M$ with $\varphi(M)=T$.
Root $T$ at the northeast corner vertex and direct all edges toward the root. This allows us to fill in the dominoes along the tree. 

Now we need to fill in the remainder. The remaining squares can be divided into several connected component $H_1,\ldots, H_l$ ( of the dual graph of the board). Note that each $H_i$ is acyclic, otherwise $H_i$ would divide $T$ into $\geq 2$ connected components. So, the all $H_i$ are trees.

Similarly, each $H_i$ intersects the boundary of the boundary of the board exactly once. Therefore each $H_i$ has a unique root. By the same method we could tile all the area.
\end{proof}
\subsection{Matrix Tree Theorem}
\begin{theorem}[Matrix-Tree Theorem]
The number of spanning trees of a graph $G$ equals
\begin{align*}
&\det(L(G)_{V(G)\backslash \{v\},V(G)\backslash \{v\}}) \text{ any }v \text{ in }V(G)\\
=&\dfrac{1}{n}\mu_1...\mu_{n-1} \hspace{1em}0,\mu_1,...,\mu_{n-1} \text{ are Laplacian eigenvalue}.
\end{align*}
\end{theorem}

\begin{proof}
(This is the second part)
We need to show that
$$\det(L(G)_{V(G)\backslash \{v\},V(G)\backslash \{v\}}) \text{ any }v \text{ in }V(G)
=\dfrac{1}{n}\mu_1...\mu_{n-1}$$.
\begin{lemma}
Let $A\in \mathbb{C}^{n\times n}$ be a matrix where eigenvalues are $0,\mu_1,\mu_2,...,\mu_{n-1}$. Then $\mu_1\mu_2...\mu_{n-1} = \sum\limits_{i=1}^n\det(A_{[n]\backslash \{i\},[n]\backslash \{i\}})$ 
\end{lemma}
\begin{proof}
Note that
\begin{align*}
\Phi_A(t) &= t(t-\mu_1)...(t-\mu_{n-1})\\
[t]\Phi_A(t) &= (-1)^{n-1}\mu_1...\mu_{n-1}\\
\text{on the other hand, } [t]\Phi_A(t) &=[t]\det(tI_n-A)\\
&=\sum\limits_{i=1}^n\det((-A)_{[n]\backslash \{i\},[n]\backslash \{i\}})\\&=(-1)^{n-1}\sum\limits_{i=1}^n\det((A)_{[n]\backslash \{i\},[n]\backslash \{i\}})
\end{align*}
\end{proof}
Apply the lemma to $A=L(G)$.
\end{proof}

\subsection{Directed Matrix Tree Theorem}
\begin{definition}
Let $D$ be a directed graph, the $Laplacian$ of $D$ $L(D)\in \mathbb{R}^{V(D)\times V(D)}$ is defined by
$$L(D)_{v,w} = \left\{\begin{aligned}
 &\text{outdeg}(v)
 -(\text{loops at }v)& \hspace{1em} \text{if }v=w\\
&-(\text{edges }v\rightarrow w)neq w
\end{aligned} \right. $$

\end{definition}
\begin{remark}
\begin{itemize}
\item In our definition, we allow $D$ with loop and multiple edges.
\item Note that $L(D)\vec{e}=0$.
\end{itemize}
\end{remark}


\begin{definition}
A rooted (directed) graph is a (directed) graph with a distinguished vertex, root. A rooted in tree is the (unique) orientation of an undirected rooted tree such that every edge points toward the root.
\end{definition}

\begin{theorem}[Tutle(1948)Directed matrix tree theorem]
Let $D$ be a directed graph and $v\in V(D)$. Then the number of spanning tree rooted at $v$ is
$$\det(L(D)_{V(D)\backslash \{v\},V(D)\backslash \{v\}})$$
\end{theorem}
\begin{proof}
(Later...)
\end{proof}

\begin{remark}
The directed matrix-tree theorem implies Kirchhoff's matrix-tree theorem:

Given $G$, let $D$ be the directed graph with $V(D)=V(G)$ and edges $v\leftrightarrows w$ for every $\{v,w\}$ of $G$.
\end{remark}

\section{Eulerian Cycles}
\begin{definition}
Let $d,n\in\mathbb{Z}_{>0}$. A $d$-ary sequence of length $n$ is a sequence of length $n$ on the alphabet $\{0,1,\ldots,d-1\}$. A $d$-ary de Bruijn sequence of order $n$ is a sequence $(a_1,\ldots,a_{d^n})\in\{0,1,\ldots,d-1\}^{d^n}$ such that the subsequences $(a_j,a_{j+1},\ldots,a_{j+n-1})$ (n-modular) are all the $d$-ary sequences of length $n$ (each appearing exactly once.
\end{definition}

\begin{example}
A binary de Bruijn sequence of order 3 is 
\[
00010111 \textrm{ and } 00011101
\]
\end{example}

Why do de Bruijn sequences exist?

How many of them are there?

How do we construct them?

Applications:
\begin{enumerate}
\item Position determination along a circular hallway.
\item Magic: Ask for the suits of any $n$ consecutive cards in a deck of size $2^n$, etc.
\end{enumerate}

Let us show that de Bruijin sequences exist, for every $d,n$. 
\begin{definition}
Define the de Bruijn graph $D_{d,n}$ as the directed graph with vertex set $\{0,\ldots,d-1\}^{n-1}$ with edges $(a_1,a_2,\ldots,a_{n-1})\to (a_2,\ldots,a_{n-1},a_n)$ for all $(a_1,\ldots,a_n)\in\{0,\ldots,d-1\}^n$.
\end{definition}

\begin{example}
$D_{2,3}$:\begin{flushleft}\end{flushleft}
\begin{center}
\begin{tikzpicture}
\tikzset{vertex/.style = {shape=circle,draw,minimum size=1.5em}}
\tikzset{edge/.style = {->,> = latex'}}
% vertices
\node[vertex] (a) at  (0,0) {00};
\node[vertex] (b) at  (3,0) {01};
\node[vertex] (c) at  (3,3) {11};
\node[vertex] (d) at  (0,3) {10};
%edges
\draw[edge] (a) to (b);
\draw[edge] (b) to (c);
\draw[edge] (c) to (d);
\draw[edge] (d) to (a);
\draw[edge] (b) to[bend left] (d);
\draw[edge] (d) to[bend left] (b);
\Loop[dist=1.5cm,dir=SOWE,labelstyle=above](a) 
\Loop[dist=1.5cm,dir=NOEA,labelstyle=above](c)  
\end{tikzpicture}
\end{center}
\end{example}

Key observations: $d$-ary de Bruijn sequences of order $n$ correspond to Eulerian cycles of $D_{d,n}$, i.e. directed cycles which use every edge (exactly once).

\begin{theorem}[Euler]
An undirected graph (without isolated vertices) is Euclian if and only if every vertex has even degree.
\end{theorem}

\end{document}