\section{Bipartite Graphs}
Recall graph is bipartite with biparts \(x,y\) \iff
\[ A(G)=\begin{bmatrix}0 & B \\ B^T & 0\end{bmatrix} \]

\begin{lemma}
Let \(B\) be an \(n\times n\) matrix such that \(B^TB\) has nonzero eigenvalues \(\lambda_1,\cdots,\lambda_r\) (with multiplicities), then the nonzero eigenvalues of 
\[ A=\begin{bmatrix}0 & B \\ B^T & 0\end{bmatrix}\in\R^{(m+n)\times(m+n)} \]
are precisely \( \sqrt{\lambda_1},-\sqrt{\lambda_1},\cdots,\sqrt{\lambda_r},-\sqrt{\lambda_r} \) (with multiplicities).
\end{lemma}
\begin{remark}
The eigenvalues of \(B^TB\) are real (from symmetry) and nonnegative since if \(B^T Bx=\lambda x\) (we assume \(||x||=1\))
\[ \lambda =\lambda x^T x= x^T (\lambda x)= x^T(B^T Bx) =(Bx)^T(Bx)=\langle Bx,Bx\rangle\geq0. \]
\end{remark}
\begin{proof}
Note that
\[ \begin{bmatrix}t I_m & -B \\ -B^T & t I_n\end{bmatrix}\begin{bmatrix} I_m & B \\ 0 & t I_n\end{bmatrix}=\begin{bmatrix}t I_m & 0 \\ -B^T & t^2 I_n-B^TB \end{bmatrix}. \]
Take determinant gives
\[ \phi_A(t) t^n=t^m\phi_{B^TB}(t^2) \]
\end{proof}

\begin{example}
Let \(K_{m,n}\) be the complete bipartite graph with biparts of size \(m\) and \(n\) and all possible edges between them, then
\[ A(K_{m,n})=\begin{bmatrix}0 & B \\ B^T & 0\end{bmatrix} \]
where \(B\) is an \(m\times n\) all-one matrix. We have
\[ \underbrace{B^T B}_{n\times n} = m J_n \]
where \(J_n\) is the \(n\times n\) all-one matrix.
Recall that \(J_n\) has exactly one nonzero eigenvalues namely \(n\), so the eigenvalues of \(K_{m,n}\) are \(\sqrt{mn},-\sqrt{mn}\) and 0 (multiplicity \(m+n-2\)).
\end{example}

\begin{example}
Let \(C_{2n}\) be a bipartite graph with adjacency matrix 
\[ A(C_{2n})=\begin{bmatrix}0 & B \\ B^T & 0\end{bmatrix} \]
where \(B\in\R^{n\times n}\) \textit{e.g.}
\[B = \begin{bmatrix}1&0&0&0&1 \\ 1&1&0&0&0 \\ 0&1&1&0&0 \\ 0&0&1&1&0 \\ 0&0&0&1&1 \end{bmatrix}\in\R^{5\times 5} \]
We can verify that
\[ B^T B=2I_n+\underbrace{A(C_n)}_{\begin{bmatrix}0&1&0&0&1 \\ 1&0&1&0&0 \\ 0&1&0&1&0 \\ 0&0&1&0&1 \\ 1&0&0&1&0 \end{bmatrix}} \]
Therefore if \(\lambda_1,\cdots,\lambda_n\) are the eigenvalues of \(C_n\), then the eigenvalues of \(C_{2n}\) are
\[ \pm\sqrt{2+\lambda_k}, \quad k=1,\cdots,n \]
This agrees with our early calculations
\[ \lambda_k=2\cos\left(\frac{2\pi k}{n} \right) \]
with identity
\[ 1+\cos(2\theta)=2\cos^2\theta \]
\end{example}


\begin{corollary}
If \(G\) is bipartite graph on \(n\) vertices then \(\lambda_n(G)=-\lambda_1(G)\).
\end{corollary}
\begin{remark}
A converse holds for connected graphs.
\end{remark}
\begin{proposition}
If \(G\) is connected on \(n\) vertices and \(\lambda_n(G)=-\lambda_1(G)\), then \(G\) is bipartite.
\end{proposition}
\begin{remark}
It's another consequence of the Perron-Frobenius theorem. We don't prove this.
\end{remark}



\section{Cartesian Products}





